\documentclass[12pt]{extarticle}
\usepackage[english, russian]{babel}
\usepackage[T2A]{fontenc}
\usepackage[utf8]{inputenc}
\usepackage[a4paper, portrait, margin=2.5cm]{geometry}
\usepackage{mathastext}
\usepackage{graphicx}
\usepackage{multirow}
\graphicspath{{images/}}
\newcommand\tab[1][1cm]{\hspace*{#1}}

\begin{document}

\section*{1}
\subsection*{а -> б}
Предположим, что не так. Значит либо существует единственный путь между u и v, тогда на этом пути существует мост (любое ребро), то есть противоречие. Либо есть хотя бы 2 пути, которые пересекаются. Тогда в этом пересечении будет мост (любое ребро из него).
\subsection*{б -> в}
Есть два пути - объединяем их - получаем цикл.
\subsection*{в -> г}
Пусть есть вершина v и ребро u-w. v и u принадлежат циклу, v и w тоже. Значит между v и u есть хотя бы 2 непересекающихся пути. Аналогично с w. Дойдём из v в u по пути, не содержащим u-w, потом пройдём по u-w, и вернёмся из w в v по пути, не содержащим u-w. Получился цикл.
\subsection*{г -> а}
Предположим, что есть мост. Пусть этот мост - ребро u-w. Тогда существует цикл, содержащий вершину v = u и ребро u-w. Тогда есть цикл, содержащий u и w, значит есть какой-то путь из u в w, который не u-w. Получается, что u-w не мост. Противоречие.

\section*{2}
($\neg$б $\Rightarrow$ $\neg$а) Пусть в графе есть две вершины, любые два пути между которыми проходят через общую вершину. Эта вершина будет точкой сочленения. \\
(б $\Leftrightarrow$ в) Цикл - это объединение двух непересекающихся путей.\\
(в $\Rightarrow$ г) Пусть v - вершина, (u, w) - ребро. По предыдущему $\exists v \rightarrow u, v \rightarrow w$. Поэтому существует цикл ($v \rightarrow u \bigcup v \rightarrow w \bigcup (u, w)$).\\
($\neg$а $\Rightarrow$ $\neg$г) Пусть v - точка сочленения, u и w ее смежные вершины. Тогда не сущ. цикла содержащего u и (v, w).

\section*{3}
Проходимся dfs по графу и находим все мосты. Удаляем их. Ответ на задачу - максимум из количества рёбер во всех компонентах связности.
\newline
В оставшихся компонентах связности не будет мостов, значит между любыми двумя вершинами есть минимум 2 пути, значит их можно поместить в один цикл. Раз вершины любые, то все рёбра в получившейся компоненте должны быть разных цветов.

\section*{4}
Находим с помощью dfs все мосты. Их надо оставить неориентированными. Все остальные рёбра можно ориентировать.
\newline
Если удалить все мосты, то в каждой компоненте связности между любыми двумя вершинами будет минимум 2 пути (иначе был бы мост, который бы удалился). Значит можно поместить вершины в ориентированный цикл. (Буквально задача G из контеста).

\section*{5}
Построим стандартный граф для 2-КНФ. Находим все циклы в получившемся графе. Проверяем каждую вершину - если вершина v есть в каком-то цикле, то она не фиксирована, потому что она принимает значение в зависимости от других переменных в этом цикле. Иначе v может принимать только одно значение, значит оно фиксированно.

\section*{7}
Инициализируем пустые множества C и I и выберем произвольную стартовую вершину v, поместив
её в C.

Далее для каждой соседней вершины u проверим, соединена ли она со всеми вершинами в
текущей клике C. Если это так, то добавим вершину u в C, а иначе отправим её в независимое
множество вершин I.

После обработки всех вершин мы можем быть уверены, что C — клика (по построению), а вот
то, что I не содержит соединённых вершин, нужно проверить. Если это так, то граф расщепляемый
и мы сразу же предоставили вариант расщепления графа.

В противном случае при таком построении граф не является расщепляемым. Однако, это значит, что и никакое другое построение не сделает граф G расщепляемым. Действительно, пусть
существует две вершины u и v из I соединённые ребром. Поскольку они не попали в клику C,
то их степени не выше порядка C. Если хотя бы одна из вершин того же порядка, что и C, то
получаем противоречие, ведь по построению такая вершина должна была оказаться в C.

Если же степени вершин ниже порядка клики C, то рассмотрим варианты построения:

1. Если начать построение из любой вершины имеющейся клики C, то расщепление не изменится, так как любая вершина из клики C сгенерирует всю ту же клику, а значит и множество
I не изменится.

2. Если начать построение из вершины I, которая оказалась соединённой с какой-то из своих
вершин, то мы быстро зайдём в тупик: пройдя к соседям этих вершин по рёбрам, связывающим I с C окажется, что вершины на концах этих рёбер со стороны C либо не связаны с
обоими вершинами I и тогда новая клика состоит всего из двух вершин, а старая лежит в I
и не является независимым множеством, либо обе вершины связаны с обеими вершинами I,
откуда следуют ещё два случая:

(a) Если обе вершины подсоединены к различным вершинам C, то тогда эта пара вершин
снова образует клику размера 2, в то время как старая клика не способна образовать
новое независимое множество.

(b) Если обе вершины подсоединены к одной вершине C, то тогда эта пара образует клику
размера

3. Но поскольку размер старой клики не меньше 3, то разбить её на независимое
множество не получится. Случай размера 3 очевиден, поскольку тогда граф "симметричен"относительно того, как начинать строить расщепление.

Таким образом, мы пришли к тому, что достаточно рассмотреть одно по-строение, чтобы сделать вывод — расщепляется граф или нет. Асимптотика такого решения будет O(n+m), поскольку
разбиение графа на C и I можно сделать за O(n + m), а проверка множества I на корректность
делается за O(n).

\end{document}