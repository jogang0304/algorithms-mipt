\documentclass[12pt]{extarticle}
\usepackage[english, russian]{babel}
\usepackage[T2A]{fontenc}
\usepackage[utf8]{inputenc}
\usepackage[a4paper, portrait, margin=2.5cm]{geometry}
\usepackage{mathastext}
\usepackage{graphicx}
\usepackage{multirow}
\graphicspath{{images/}}
\newcommand\tab[1][1cm]{\hspace*{#1}}

\begin{document}
\section*{1}


\section*{4}
Не умаляя общности будем считать, что слева стоит 0, а справа 1\newline
Бинпоиск: \newline
Смотрим на центральные два элемента. Если они оба 0, то переходим на правую половину. Если оба 1, переходим на левую. И так далее, пока не встретим два различных элемента.\newline
Когда мы попадаем на два нуля и переходим на правую половину, когда-то мы точно встретим пару различных элемментов, потому что где-то справа есть хотя бы одна единица.









\end{document}