\documentclass[12pt]{extarticle}
\usepackage[english, russian]{babel}
\usepackage[T2A]{fontenc}
\usepackage[utf8]{inputenc}
\usepackage[a4paper, portrait, margin=2.5cm]{geometry}
\usepackage{mathastext}
\usepackage{graphicx}
\usepackage{multirow}
\graphicspath{{images/}}
\newcommand\tab[1][1cm]{\hspace*{#1}}

\begin{document}
\section*{1}
$T(n) = 3T(\sqrt{n}) + log_2 n$ \newline
$n = 2^k;\ k = log_2 n$ \newline
$T(2^k) = 3T(2^{\frac{k}{2}}) + k$ \newline
Аргумент в T влияет только на количество вызовов функции. На время выполнения влияет слогаемое вне T. Поэтому можно перейти от $T(2^k) = 3T(2^{\frac{k}{2}})$ к $T(k) = 3T(\frac{k}{2})$ \newline
$T(k) = 3T(\frac{k}{2}) + k$ \newline
По мастер теореме: \newline
$a = 3;\ b = 2;\ f(k) = O(k^{log_2 3 - \epsilon}) = O(k^1)$ \newline
$\epsilon > 0 \rightarrow$ первый случай \newline
$T(k) = \theta(k^{log_2 3})$ \newline
Так как k как и n уменьшается до примерно 1, то T(k) = T(n). Например если $T(k) = O(k)$, то $T(n) = O(logn) = O(k)$\newline
$O(n) = \theta(log^{log_2 3}n)$

\section*{2}
$T(n) = 2T(\frac{n}{2}) + nlog_2 n$ \newline
$T(n) = nlog_2 n + 2\frac{n}{2}log_2 \frac{n}{2} + 4T(\frac{n}{4})$\newline
$T(n) = nlog_2 n + n(log_2 n - 1) + n(log_2 n - 2) + \dots$\newline
$T(n) = log_2 n (nlog_2 n) - (0 + 1 + 2 + \dots + (n-1))$\newline
$T(n) = n log_2^2 n - \frac{log_2 n (n-1)}{2}$\newline
$T(n) = O(nlog_2^2 n - nlog_2 n) = O(nlog_2^2 n)$

\section*{3}
Как в задаче на добавление числа ко всем элементам на отрезке массива.\newline
Нужно создать дополнительные массивы для b и для d. Они хранят разницу b и d между двумя соседними элементами. При выводе массива к элементу нужно добавлять нужное b, а также какое-то накопленное значение (на каждом шаге к нему прибавляется d). Иногда это накопленное значение нужно резко менять на большую величину, для этого можно создать третий массив (или записывать такие резкие изменения в массив для b).
\section*{4}
Не умаляя общности будем считать, что слева стоит 0, а справа 1\newline
Бинпоиск: \newline
Смотрим на центральные два элемента. Если они оба 0, то переходим на правую половину. Если оба 1, переходим на левую. И так далее, пока не встретим два различных элемента.\newline
Когда мы попадаем на два нуля и переходим на правую половину, когда-то мы точно встретим пару различных элемментов, потому что где-то справа есть хотя бы одна единица.

\section*{5}
Проходимся по массиву. На каждом шаге обновляем максимум из элементов a от a0 до ai, а также максимум суммы = max(max(a0:ai) + bi, curAnswer).

\section*{6}
Проверяем рекурсивно. Для [a0, an] считаем b = (n - i + 1 + x) - sum(ai:an). Если an+1 <= b + 1, то b = b + 1 - ab+1. Иначе ответ = нет.

\section*{7}
Два индекса: i на начало первого массива и j на конец второго.\newline
Если сумма элементов на этих индексах равна k, то ans++, изменяем один любой индекс.
Если сумма меньше k, то i++.
Если сумма больше k, то j--.









\end{document}