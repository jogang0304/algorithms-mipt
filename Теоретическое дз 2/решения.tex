\documentclass[12pt]{extarticle}
\usepackage[T2A]{fontenc}
\usepackage[utf8]{inputenc}
\usepackage[english, russian]{babel}
\usepackage[a4paper, portrait, margin=2.4cm]{geometry}
\usepackage{mathastext}
\usepackage{amsmath}
\usepackage{amssymb}
\usepackage{graphicx}
\usepackage{multirow}
\usepackage{ragged2e}
\usepackage{wrapfig}
\pagenumbering{arabic}
\graphicspath{{images/}}
\newcommand\tab[1][1cm]{\hspace*{#1}}

\begin{document}

\section*{5}
Делаем обычную кучу на минимум, в добавок к которой будем хранить значение наибольшего элемента в ней.
Insert, getMin и extractMin работают за O(logn).
Insert может обновить наибольший элемент (сравнение и обновление - O(1)). ExtractMin обновляет наибольший элемент только если к моменту вызова extractMin в куче оставался только один элемент (после его удаления maxElement = $-\infty$). Поэтому обновлений почти не будет. GetMax работает за O(1).

\end{document}