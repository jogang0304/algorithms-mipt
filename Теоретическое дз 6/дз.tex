\documentclass{extarticle}
\usepackage[T2A]{fontenc}
\usepackage[utf8]{inputenc}
\usepackage[english,russian]{babel}
\usepackage{geometry}
\geometry{a4paper, margin=2cm}


\begin{document}

\section*{1}
Очевидно, что первый гвоздик нужно соединить со вторым.
Третий гвоздик можно соединить со вторым или с четвёртым. Будем хранить два массива a и b. В массиве a будем хранить длину ниток, если мы соединили гвоздик i с i-1. В массиве b храним длину ниток, если не соединили. i от 1 до n. a[1] = none, b[1] = 0, a[2] = l[1, 2], b[2] = none. Далее, a[i] = min(a[i-1], b[i-1]) + l[i-1, i], b[i] = a[i-1]. Ответом будет a[n].

\section*{2}


\end{document}
