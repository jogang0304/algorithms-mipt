\documentclass{extarticle}
\usepackage[T2A]{fontenc}
\usepackage[utf8]{inputenc}
\usepackage[english,russian]{babel}
\usepackage{geometry}
\usepackage{listings}
\geometry{a4paper, margin=2cm}


\begin{document}

\section*{1}
Отсортируем гвозди по возрастанию координаты.
\newline
Очевидно, что первый гвоздик нужно соединить со вторым.
\newline
Третий гвоздик можно соединить со вторым или с четвёртым. Будем хранить два массива a и b. В массиве a будем хранить длину ниток, если мы соединили гвоздик i с i-1. В массиве b храним длину ниток, если не соединили. i от 1 до n. a[1] = none, b[1] = 0, a[2] = l[1, 2], b[2] = none. Далее, a[i] = min(a[i-1], b[i-1]) + l[i-1, i], b[i] = a[i-1]. Ответом будет a[n].
\newline
Асимптотика O(n log n) из-за сортировки.

\section*{2}
Сделаем двухмерный массив dp.
\newline
В dp[i][j], i < j хранится сколько нужно минимально переправ для перевоза грузов с i по j, и сколько свободного места есть в последней переправе.
\newline
dp[i][i] = $\{1, t - w[i]\}$
\newline
dp[i][j] =
\begin{lstlisting}
    if (dp[i + 1][j].first < dp[i][j - 1].first) {
        if (dp[i + 1][j].second >= w[i]) {
            dp[i][j] = {dp[i + 1][j].first, dp[i + 1][j].second - w[i]}
        } else {
                dp[i][j] = {dp[i + 1][j].first + 1, t - w[i]}
        }
    } else if (dp[i + 1][j].first > dp[i][j - 1].first) {
        if (dp[i][j - 1].second >= w[j]) {
            dp[i][j] = {dp[i][j - 1].first, dp[i][j - 1].second - w[j]}
        } else {
            dp[i][j] = {dp[i][j - 1].first + 1, t - w[j]}
        }
    } else {
        if (dp[i + 1][j].second >= dp[i][j - 1].second) {
            if (dp[i + 1][j].second >= w[i]) {
                dp[i][j] = {dp[i + 1][j].first, dp[i + 1][j].second - w[i]}
            } else {
                dp[i][j] = {dp[i + 1][j].first + 1, t - w[i]}
            }
        } else {
            if (dp[i][j - 1].second >= w[j]) {
                dp[i][j] = {dp[i][j - 1].first, dp[i][j - 1].second - w[j]}
            } else {
                dp[i][j] = {dp[i][j - 1].first + 1, t - w[j]}
            }
        }
    }
\end{lstlisting}
Ответом будет dp[1][n].first.

\section*{3}

Разобьём элементы на группы по $\sqrt{n}$ элементов. Решим задачу о рюкзаке для каждого из них за $O(W\sqrt{n} \cdot \sqrt{n}) = O(Wn)$ времени и $O(W\sqrt{n})$ памяти. На каждую группу тратится $O(W\sqrt{n})$ памяти, и надо хранить верхний слой каждой из уже подсчитанных групп. Длина слоя W, количество групп $\sqrt{n}$. Поэтому суммарно памяти используется $O(W\sqrt{n})$.


\end{document}