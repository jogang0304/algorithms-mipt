\documentclass[12pt]{extarticle}
\usepackage[english, russian]{babel}
\usepackage[T2A]{fontenc}
\usepackage[utf8]{inputenc}
\usepackage[a4paper, portrait, margin=2.5cm]{geometry}
\usepackage{mathastext}
\usepackage{graphicx}
\usepackage{multirow}
\graphicspath{{images/}}
\newcommand\tab[1][1cm]{\hspace*{#1}}

\begin{document}

\section*{1}
С помощью очереди с поддержкой минимума найдём $\frac{1}{8}$ вершин с максимальным количеством исходящих из них рёбер за O(n). Удалим их и инцидентные им рёбра. Запустим на оставшемся графе алгоритм Краскала или Прима для нахождения минимального остовного дерева. Рёбер и вершин в новом графе меньше, чем m и n, поэтому асимптотику можно упростить до O(m + n).

\section*{2}
Идея:\\
Сортируем левые вершины по правому ограничению.\\
Перебираем левые вершины:\\
Если не задействована, ищем подходящую правую по ограничению и тоже не задействованную.\\
Если нашли, добавляем пару в паросочетание и отмечаем обе вершины как занятые.

\section*{3}
Лемма: Каждая вершина в ациклическом транзитивном орграфе принадлежит ровно одному максимальному независимому множеству. (Доказательство по индукции).\\
Теорема: Наименьшее количество независимых множеств, на которые можно разбить все вершины ациклического транзитивного орграфа $G$, равно размеру самого длинного пути в $G$.
\newline
1. Пусть $p$ - самый длинный путь в $G$.\\
2. Разделим вершины $G$ на $p$ множеств $M_1, \dots, M_p$ таким образом, чтобы $M_i$ содержало вершины исключительно с пути длины $i$.\\
3. Каждое $M_i$ является независимым множеством: Если бы две вершины $v$ и $w$ из $M_i$ доминировали друг друга, то существовал бы путь длиной не меньше $p$, противоречие.\\
4. Невозможно разбить $G$ на меньше, чем $p$ множеств:\\
- Предположим, что существует разбиение на $k$ множеств ($k < p$). \\
- Выберем $M_i$ и вершину $v$ из него. \\
- Выберем $w$ с самого длинного пути, не содержащего $v$. \\
- $w$ не может быть в $M_i$, но и ни в одном другом множестве, так как это противоречит независимости или длине пути, ведущему к $w$. - Противоречие.\\

Таким образом, размер самого длинного пути определяет минимальное количество независимых множеств для разбиения вершин $G$.

\section*{4}
\subsection*{а}
Необходимость:

- Если стартовая вершина $v$ не во всех паросочетаниях, существует паросочетание $M$, где $v$ не насыщена.

- Второй игрок ходит по ребрам из $M$, пока не дойдет до свободной вершины.

- Первый игрок не сможет ответить ходом, так как все ребра, ведущие обратно в $v$, уже использованы.
\newline
Достаточность:

- Если $v$ во всех паросочетаниях, то из нее выходит не меньше ребер, чем входит.

- На любой ход второго игрока первый игрок может ответить ходом по одному из ребер, ведущих из $v$ в каком-либо максимальном паросочетании.

\subsection*{б}

1) Алгоритм O($n^2 m$):

- Для каждой вершины $v$ строим обратный граф $G_v$.

- Найти максимальное паросочетание в каждом $G_v$.

- $v$ выигрышная, если она во всех паросочетаниях $G_v$.
\newline
2) Алгоритм O(nm):

- Считать степени захода $d_v^{in}$ и исхода $d_v^{out}$ для каждой вершины.

- $v$ выигрышная **тогда и только тогда**, когда $d_v^{in} \ge d_v^{out}$.

\section*{5}
Для нахождения множества непересекающихся циклов, покрывающих все вершины, можно применить алгоритм поиска циклов в графе, который не пересекаются, но покрывают все вершины. Раздвоив вершины графа, можно использовать алгоритм поиска Эйлерова цикла.

\section*{6}
Алгоритм:
\begin{enumerate}
    \item  Поиск в ширину: \begin{itemize}
        \item Выполнить поиск в ширину из одной из долей графа.
        \item Отметьте все достигнутые вершины другой доли как "посещенные".
    \end{itemize}
    \item Массивы насыщения: Создайте два массива $sat_u$ и $sat_v$ для долей, где $sat_u[i]$ - количество рёбер, соединяющих вершину i с "посещенными" вершинами.
    \item Анализ рёбер: Для каждого ребра (u, v): \begin{itemize}
        \item $inc_u = sat_u$[u] + 1 (если v не отмечена)
        \item $inc_v = sat_v$[v] + 1 (если u не отмечена)
        \item Может быть в МП: $inc_u == n_u$ и $inc_v == n_v$ (степени вершин)
        \item Должно быть в МП: $sat_u[u] == n_u$ или $sat_v[v] == n_v$
        \item Не может быть в МП: Остальные случаи
    \end{itemize}
\end{enumerate}
    
\section*{7}
\begin{enumerate}
    \item Проверка условий:
        \begin{enumerate}
            \item Если все клетки заблокированы: ответ НЕТ
            \item Если количество свободных клеток не кратно 3: НЕТ
            \item Пройтись по строкам и столбцам:
                \begin{enumerate}
                    \item Если количество свободных клеток в строке/столбце не кратно 3: НЕТ
                \end{enumerate}
        \end{enumerate}
    \item Поиск циклов (алгоритм Эйлера):
        \begin{enumerate}
            \item Построить граф:\\
                * Вершины - свободные клетки.\\
                * Ребра - общие стороны соседних свободных клеток.
            \item Если алгоритм Эйлера не найдет циклов: НЕТ
            \item Если найдено недостаточно циклов, чтобы покрыть все клетки: НЕТ
            \item Иначе: ДА
        \end{enumerate}
\end{enumerate}

Асимптотика:\\
* $O(n + m)$ - проверка условий.\\
* $O(nm)$ - поиск циклов.

\section*{8}
Построение графа:\\
Создайте граф, где вершина 1 (source) соединена с каждой "кочкой" (вершинами, которые коровки могут посетить) ребром с пропускной способностью 1.\\
Каждая "кочка" соединена с вершиной n (sink) ребром с пропускной способностью 1.\\
Между всеми "нестабильными" вершинами (кочками) устанавливаются рёбра с бесконечной (или большой) пропускной способностью.\\
Нахождение максимального потока:\\
Используйте алгоритм поиска максимального потока в построенном графе. Эффективным выбором может быть алгоритм Эдмондса-Карпа, который работает за время $O(V*E^2)$, где V - количество вершин, E - количество рёбер.

\section*{9}
Чтобы решить эту проблему, мы можем использовать динамическое программирование, чтобы максимизировать удовольствие Васи в течение нескольких дней при соблюдении заданных ограничений. Мы определим таблицу динамического программирования, чтобы отслеживать максимальное удовольствие, достижимое для каждого дня с учетом ограничений.
\end{document}